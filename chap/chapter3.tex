Dieses Dokument über die LaTeX-Vorlage wurde seinerseits mit der Protokollvorlage erstellt (nur das Deckblatt wurde weggelassen).

\section{Erläuterungen zum Aufbau der Vorlage}
Das Praktikumsprotokoll soll auf das spätere Verfassen eines Textes als Wissenschaftler vorbereiten. Klar ist, dass das Praktikum als Lehrveranstaltung jedoch zugleich den Anspruch einer Prüfung an das Protokoll erhebt und auch eine gewisse praktische Arbeitsweise fordert (nur einige Tage Bearbeitungszeit). Richtig ist, dass man überlicherweise bestimmte Dinge, die in einem Praktikumsprotokoll gefordert werden, nicht in einem Bericht über den Aufbau eines tatsächlichen Experiments oder einer tatsächlichen wissenschaftlichen Arbeit zu finden sind. Klar ist z.B., dass das Verfassen zweier theoretischer Vorbereitungen in ein Thema innerhalb eines Dokuments vollkommen unüblich ist und dennoch für das Praktikumsprotokoll verlangt wird. Es gibt auch andere stilistische Einschränkungen. Überlicherweise unterteilt man wissenschaftliche Berichte in Kapitel, bei denen jedes Kapitel auf einer ungeraden Seite beginnt. Bei einem 20-seitigen Praktiumsbericht handelt es sich hingegen schlicht um Papierverschwendung. Der Nutzer der LaTeX-Vorlage hat hier die freie Wahl, ob er danach verfahren möchte oder nicht. Es gibt bspw. auch eine Funktion, bei der die Seitennummerierung in einem Kapitel stets von neuem beginnt. Dies ist nützlich, wenn mehrere Personen unabhängig an einem Dokument schreiben möchten, ohne dass zum Schluss Chaos bei der Seitennummerierung herrscht. Bei der Vorlage bleibt es also den Teilnehmern des Praktiums selbst überlassen, ob sie ihr Dokument nach strengen stilistischen Regeln oder eher praktisch orientiert aufbauen möchten. Auf das Erstellen eines \glqq richtigen\grqq\ Deckblatts neben dem ausfüllbaren Deckblatt für die Organisation in der Lehrveranstaltung verzichtet diese Vorlage gänzlich, wieder aus Sparsamkeit an Resourcen. Studenten, die besonderen Wert auf Vollständigkeit und Ordnung legen, können ein zusätzliches eigenes Deckblatt einrichten. Weitere Details zu den Optionen der Vorlage finden sich im entsprechenden Kapitel dieses Dokuments.



\section{Die TeXnische Umsetzung}
Die Datei \verb|main.tex| muss mit PDFLaTeX kompiliert werden, um die druckbare PDF-Datei \verb|main.pdf| zu erstellen. In der Datei \verb|main.tex| werden alle Einstellungen am Dokument und die eingebundenen Kapitel definiert. Die Datei basiert auf einer speziell für das Praktikum angelegten Dokumentenklasse, die sich im Ordner \verb|misc/| mit dem Namen \verb|protokollclass.cls| befindet. Diese definiert das Layout des Protokolls.

Der erste größere Block am Anfang der Datei besteht nur aus der Dokumentenklasse (\verb|\documentclass|) und einer Aufzählung der zum Dokumente gehörigen Dateien.

Der zweite Block mit der Überschrift \verb|Settings for word separation| dient dazu, LaTeX zu vermitteln, wie Wörter zu trennen sind, die LaTeX von Hause aus nicht kennt. Meist kommt LaTeX auch ohne dies klar, aber hin und wieder ergeben sich unschöne Probleme bei sehr langen Wörtern, die man lieber vermeiden möchte.

Der dritte Block beinhaltet einige auskommentierte Befehle, die bei Bedarf eingeschaltet werden können. Diese Befehle sind in der Datei kommentiert. Die letzten beiden Optionen zur Seitenzahlnummerierung sind nützlich, falls die Praktikumspartner im P1 ihre Vorbereitungen unabhängig voneinander schreiben möchten. Grundsätzlich bietet es sich an, zunächst nur die Vorbereitungen zu schreiben und nur die nötigen Seiten für den Versuchstag auszudrucken und anschließend das Kapitel zur Auswertung zu ergänzen und den Rest auszudrucken, wenn alles fertig ist. Wenn sich die beiden Partner nicht richtig absprechen und dafür sorgen, dass die Vorbereitungen von einer Person in ein LaTeX Dokument eingebunden werden, endet dies darin, dass die Seitenzahlen der einzelnen Vorbereitungen stets bei 1 beginnt. Man kann dann für das ganze Protokoll einfach direkt die Seitenzahl zu Beginn eines jeden neuen Kapitels auf 1 zurücksetzen. Um Warnungen von LaTeX zu unterdrücken und dem Leser eine bessere Übersicht zu geben, besteht dazu die Möglichkeit, die Kapitelnummer vor die Seitenzahl auf jede Seite drucken zu lassen.

Falls ein Protokoll auf Englisch verfasst werden soll, können vordefinierte Begriffe wie Literaturverzeichnis (Engl.: Bibliography) oder Kapitel (Engl.: Chapter) auf eine gewählte Sprache eingestellt werden. Diese Sprache wird auch im Literaturverzeichnis verwendet.

Nach dem Befehl \verb|\begin{document}| wird der eigentliche Inhalt des Dokuments festgelegt. Die vordefinierten Befehle \verb|\titelseiteshorttoc| und \verb|\titelseitelongtoc| erzeugen Titelseite und Inhaltsverzeichnis in einem. Bei der langen Version wird ein Seitenumbruch eingefügt. Überlicherweise beginnt das Literaturverzeichnis eines Buches, eines Berichtes oder ähnlicher Dokumente auf einer ungeraden Seite. Da nun Praktikumsprotokolle vielleicht nicht den höchsten Anspruch erheben müssen und bei den vielen Protokollen pro Jahr einiges an Papier gespart werden kann, ist es vielleicht empfehlenswert sich für die kürzere Version zu entscheiden, auch wenn es normalerweise stilistisch anders gehalten wird.

Mit dem Befehl \verb|\MainMatter| werden Seitennummerierung, Layout, etc. auf den Beginn des eigentlichen Inhalts des Dokuments eingestellt. Es folgen die verschiedenen Kapitel, die mit \verb|\chapter{Kapitelname}| definiert werden. Der Inhalt wird aus \verb|.tex|-Dateien mit Hilfe von \verb|\input{...}| eingelesen. Auch hier können Kapitel wieder auf geraden sowie auf ungeraden Seiten eröffnen. Wer damit nicht zufrieden ist, kann mit \verb|\cleardoublepage| nach dem \verb|\input|-Befehl nachhelfen.

Der Anhang dürfte für die meisten Protokolle nicht notwendig sein, man kann ihn jedoch sehr leicht erzeugen. \verb|\Appendix| stellt wieder das Layout um und die nachfolgenden Befehle erzeugen ein unnummeriertes Kapitel \textit{Anhang} mit, welches wieder über \verb|\input{...}| eingebunden wird.

Mit dem Literaturverzeichnis tun sich viele Studenten im Praktikum schwer, da das meiste ohnehin nur aus der Vorbereitungsmappe stammt. Dennoch stellt die Vorlage sowohl ein einfaches \verb|\TheBibliography|- sowie ein umfangreiches Bibtex-Literaturverzeichnis zur Verfügung. Der Befehl \verb|\TheBibliography| dient wieder zum Verstellen des Layouts. Danach kann man entweder mithilfe von Bibtex über den Befehl \verb|\bibliography{file.bib}| ein Literaturverzeichnis erzeugen. Die Datenbank muss dabei im Bibtex-Format in der Datei \verb|file.bib| vorliegen. Ein Beispiel befindet sich in der Datei \verb|misc/lit.bib|. Das Layout des Literaturverzeichnisses legt man über den Befehl \verb|\bibliographystyle{style}| fest. In dieser Vorlage wird der Stil \verb|dcbib| verwendet. Es können auch alle weiteren kompatiblen Stile verwendet werden. Eine Einführung in die verschiednen Zitierstile erfolgt  hier nicht. Alle kompatiblen Stile findet man im Dokument \textit{BibTeX styles catalogue for LyX}: \url{http://bradlug.co.uk/wp-content/uploads/2009/02/BibTeX_styles_catalogue_for_LyX.pdf}. Diese laufen nicht nur mit Bibtex, sondern auch mit der internationalisierten Version, Babelbib. Alternativ zum Bibtex-Literaturverzeichnis kann man ein simpleres Literaturverzeichnis mit dem Befehl \verb|\TheBibliography| erstellen, welcher von Hause aus mit LaTeX kommt. Die Syntax ist im Dokument erläutert.

\section{Weitere Dateien}
Die Daten zum Dokument, die auf dem Deckblatt angezeigt und in den Metadaten der PDF-Datei gespeichert werden, können über die Datei \verb|misc/data.tex| verändert werden. Einige Kurzbefehle (z.B. \verb|\element{C}{12}{6}| für \element{C}{12}{6}) wurden in der Datei \verb|cmds.tex| angelegt. Weitere können hier nach Belieben hinzugefügt werden. Die Datei \verb|titelpage.tex| selbst sollte aus organisatorischen Gründen nach Möglichkeit nicht bearbeitet werden, damit das Layout der Titelseite für alle abgegebenen Protokolle der verschiedenen Gruppen gleich bleibt. Die Inhalte können, wie erwähnt, über die Einträge in der Datei \verb|data.tex| angepasst werden.

Bilder und Tabellen können natürlich von überall aus eingebunden werden, aber es empfiehlt sich aus Gründen der Übersicht, für diese Dateien ebenfalls Unterordner anzulegen. Häufig verwendet man die Bezeichnungen \verb|fig| für den Unterordner mit den Abbildungen und \verb|tab| für den Unterordner mit den Tabellen.