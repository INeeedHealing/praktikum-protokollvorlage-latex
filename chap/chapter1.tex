Das nachfolgende Dokumente enthält Erläuterungen und Details zur LaTeX-Vorlage für Praktikumsprotokolle sowie einige allgemeine Ratschläge zum technischen Verfassen von Praktikumsprotokollen. Es handelt sich dabei um Protokolle für die physikalischen Anfängerpraktika der Fakultät für Physik des KIT. Das Dokument richtet sich an Studierende des KIT, die am Praktikum teilnehmen. Die Vorlage wird von der Fachschaft für Physik bereit gestellt.

Wissenschaftliche Texte sollten einige Anforderungen erfüllen. Beim Einhalten dieser Anforderungen stellen sich dem Studenten zwei wesentliche Probleme. Zum Ersten muss er die nötige Kenntnis darüber besitzen, welche Anforderungen wie zu erfüllen sind. Er muss also die formellen Regeln kennen, die beim wissenschaftlichen Schreiben einzuhalten sind. Auf dieses Problem wird hier nicht eingegangen. Entsprechende Kurse des HoC zum wissenschaftlichen Schreiben und Präsentieren sind für das Erlernen dieser Formalien empfehlenswert. Zum Zweiten jedoch muss der Student wissen, wie diese Regeln technisch, d.h.~beim Erstellen eines Dokuments am Computer, umzusetzen sind. Um die Studenten bei dem Umgang mit diesem Problem zu unterstützen, steht diese LaTeX-Vorlage zur Verfügung.

Diese Vorlage setzt die Verwendung des Textsatzsystems LaTeX voraus. Wer sich gegen LaTeX entscheidet, braucht ab dieser Stelle nicht weiter zu lesen. Das Erstellen des Praktikumsprotokolls ohne LaTeX ist durchaus möglich. Mehr Details dazu im nächsten Kapitel.