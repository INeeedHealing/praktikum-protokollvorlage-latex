\section{Allgemeine Befehlsreferenzen und Dokumentationen}
Benötigt man Hilfe zu einem bestimmten Befehl oder zu einem bestimmten Thema, wird man oft hier fündig:

\begin{itemize}
 \item \url{https://en.wikibooks.org/wiki/LaTeX}
 \item \url{https://de.wikibooks.org/wiki/LaTeX-Kompendium},
 \item \url{https://de.wikipedia.org/wiki/Hilfe:TeX},
 \item \url{http://www.ctan.org/tex-archive/info/lshort/english/lshort.pdf}.
\end{itemize}

Möchte man Hilfe zu einem bestimmten Thema erhalten, so kann man unter \url{https://tex.stackexchange.com/} sehr gezielt suchen.

Sucht man nach dem Befehl für ein ganz bestimmtes Zeichen, ist folgendes Dokument nützlich: \url{www.ctan.org/tex-archive/info/symbols/comprehensive/symbols-a4.pdf}.

Das Setzen von Größen und Einheiten will gekonnt sein, ist aber mit dem \verb|siunitx|-Paket sehr leicht: \url{http://mirrors.ctan.org/macros/latex/contrib/siunitx/siunitx.pdf}.

\section{Einige nützliche Beispiele}
\subsection{Zitate}
Zitate erstellt man mit dem Befehl \verb|\cite{key}|, wobei \verb|key| die Kurzbezeichnung des Literaturverzeichnisses ist. Das Ergebnis ist dann z.B.: \cite{EKS07}. Möchte man ganz gezielt auf ein Detail einer Quelle verweisen, so lässt sich dies über \verb|\cite[optionales]{key}| bewerkstelligen. Dies erzeugt dann: \cite[S. 42]{EKS07}.

\subsection{Anmerkungen}
Möchte man sich eine Erinnerung anlegen, was später noch erledigt werden muss, kann der von der Vorlage definierte Befehl \verb|\todo{Anmerkung}| nützlich sein. Als Beispiel:

Der Wert für die gemessene Magnetisierung $M$ des Eisenkerns nach wiederholter Magnetisierung und Entmagnetisierung beträgt \todo{Wert} bei \SI{25}{\celsius}. Diesen Werte und weitere Vergleichswerte anderer Quellen befinden sich auch in Tab. \todo{Verweis einf.}.

Hilfreich kann auch die Umgebung \verb|\begin{deprecated}...\end{deprecated}| sein, mit der man ganze Absätze, die man im Dokument nicht mehr haben will, aber zunächst noch (vielleicht als Platzhalter) aufbewahren möchte, geeignet kennzeichnen kann. Als Beispiel:

\begin{deprecated}
Dieser Text ist noch da, aber er ist eigentlich nicht sehr sinnvoll und sollte besser ersetzt oder gründlich überarbeitet werden.
\end{deprecated}
