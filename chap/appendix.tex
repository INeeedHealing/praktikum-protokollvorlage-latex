\section{Modell zur Berechnung des Zerfallsverhältnisses}
Bei chemischen Reaktion wird die Veränderung der Konzentrationen beteiligter Stoffe sowie deren Reaktionsgeschwindigkeiten mithilfe von Systemen von Differenzialgleichungen beschrieben. Es treten oft so genannte Reaktionen \glqq erster Ordnung\grqq auf, welche vom Prinzip dem Zerfall radioaktiver Präparate gleich kommen. Für eine Reaktion eines Stoffes $\mathrm A$ zu $\mathrm B$ mit Ratenkonstante $k_\mathrm B$ und Folgereaktion $\mathrm B$ zu $\mathrm C$ mit Ratenkonstante $k_\mathrm C$,
\begin{equation}
\mathrm{A} \xrightarrow{k_{\mathrm{B}}} \mathrm{B} \xrightarrow{k_{\mathrm{C}}} \mathrm{C},
\end{equation}
gelten die Differenzialgleichungen
\begin{align}
\frac{d[\mathrm A]}{dt} &= -k_{\mathrm{B}} [\mathrm A],\\
\frac{d[\mathrm B]}{dt} &= -k_{\mathrm{C}} [\mathrm B] + k_{\mathrm{B}} [\mathrm A]\quad\text{und}\nonumber\\
\frac{d[\mathrm C]}{dt} &= k_{\mathrm{C}} [\mathrm B] \nonumber
\end{align}
Als Lösung erhält man für die Konzentrationen $[\mathrm A]$ und $[\mathrm B]$:
\begin{align}
[\mathrm A] &= [\mathrm A]_0 e^{-k_{\mathrm{B}} t},\\
[\mathrm B] &= \frac{k_{\mathrm{B}} [\mathrm A]_0}{k_{\mathrm{C}} - k_{\mathrm{B}}} \left(e^{-k_{\mathrm{B}} t} - e^{-k_{\mathrm{C}} t}\right),\nonumber\\
[\mathrm C] &= [\mathrm A]_0 \left( 1 + \frac{k_{\mathrm{B}} e^{-k_{\mathrm{C}} t} - k_{\mathrm{C}} e^{-k_{\mathrm{B}} t}}{k_{\mathrm{C}} - k_{\mathrm{B}}} \right).\qquad\cite{EW11}\nonumber
\end{align}
Dieses Modell lässt sich nun einheitlich auf den Zerfall der beiden Präparate übertragen. Man identifiziert $[\mathrm A]$ mit der Zahl der $\ce{_{38}^{90} Sr}$-, $[\mathrm B]$ mit der der $\ce{_{39}^{90} Y}$- und $[\mathrm C]$ mit der der $\ce{_{40}^{90} Zr}$-Atome. Wie aus der Lösung von $[\mathrm A]$ schon deutlich wird, erhält man die Raten-Konstanten aus den Halbwertszeiten,
\begin{equation}
k = \frac{\log(2)}{T_{\nicefrac{1}{2}}}\label{eq:iii_3_umrechnung},
\end{equation}
wobei $T_{\nicefrac{1}{2}}$ die Halbwertszeit des jeweiligen Zerfalls sei (im Folgenden sei außerdem $k_{\mathrm{Sr}} = k_{\mathrm{B}}$, $k_{\mathrm{Y}} = k_{\mathrm{C}}$ und $N_{\mathrm{Sr},0} = [\mathrm A]_0$). Man berechnet nun die zeitlichen Ableitungen:
\begin{align}
\dot{N_{\mathrm{Sr}}}(t) &= N_{\mathrm{Sr},0} \cdot (-k_{\mathrm{Sr}}) e^{-k_{\mathrm{Sr}} t}\\
\dot{N_{\mathrm{Zr}}}(t) &= -N_{\mathrm{Sr},0} \cdot k_{\mathrm{Sr}} \cdot k_{\mathrm{Y}} \frac{e^{-k_{\mathrm{Y}} t} - e^{-k_{\mathrm{Sr}} t}}{k_{\mathrm{Y}} - k_{\mathrm{Sr}}}.
\end{align}
Gesucht ist das Verhältnis der Zerfälle pro Zeiteinheit. Die Zahl der $\ce{_{39}^{90} Y}$-Zerfälle ist gleich dem Negativen der zeitlichen Steigung der Zahl an $\ce{_{40}^{90} Zr}$-Atomen. Somit folgt für das Verhältnis:
\begin{align}
v = \frac{-\dot{N_{\mathrm{Zr}}}(t)}{\dot{N_{\mathrm{Sr}}}(t)} &= \frac{k_{\mathrm{Y}}}{k_{\mathrm{Y}} - k_{\mathrm{Sr}}} \left(1 - e^{(k_{\mathrm{Sr}} - k_{\mathrm{Y}})t}\right)\\
\intertext{Man verwendet dann, dass $T_{\nicefrac{1}{2},Y} \ll T_{\nicefrac{1}{2},Sr}$ und somit $k_{\mathrm{Y}} \gg k_{\mathrm{Sr}}$ und findet so:}
				     &\approx \frac{k_{\mathrm{Y}}}{k_{\mathrm{Y}} - k_{\mathrm{Sr}}} \left(1 - e^{-k_{\mathrm{Y}}t}\right).\\
\intertext{Die Konstanten sind nach Gl. \eqref{eq:iii_3_umrechnung} $k_{\mathrm{Sr}} = \SI{7,63E-10}{1/s}$ und $k_{\mathrm{Y}} = \SI{3,01E-6}{1/s}$. Setzt man $t/T_{\nicefrac{1}{2}}$ beispielsweise $100$, so findet man den $2^{\nicefrac{t}{T_{\nicefrac{1}{2}}}}$-Term in der Größenordnung von $10^{-30}$ (was weniger als einem Jahr entspräche, also recht sinnvoll wäre - zur Erinnerung: $t$ ist die Zeit nach der Anfangsbedingung $N_{\mathrm{Y}} = N_{\mathrm{Zr}}=0$). Damit lässt sich der Term mit dem Exponenten ebenfalls vernachlässigen und man findet:}
				     &\approx \frac{k_{\mathrm{Y}}}{k_{\mathrm{Y}} - k_{\mathrm{Sr}}} \approx \frac{k_{\mathrm{Y}}}{k_{\mathrm{Y}}} = 1.
\end{align}
