%% -----------------------
%% |    Abbreviations    |
%% -----------------------
\newcommand{\op}[1]{\operatorname{#1}}			% to write operators that are not predefined; it's just an abbrev. for the long cmd
\newcommand{\arr}[2]{\begin{array}{#1}#2\end{array}}	% to create arrays. very useful in math environment

\renewcommand{\d}{\ensuremath{\text{d}}}		% as the differential operator, e.g. \frac{\d x}{\d t}
\newcommand{\NN}{\mathbb{N}}					% or change it to \mathbbm and include pkg bbm if you like it better
\newcommand{\RR}{\mathbb{R}}
\newcommand{\CC}{\mathbb{C}}

\newcommand{\pdb}[2]{\frac{\partial #1}{\partial #2}} % partial derivative


%% -----------------------------------
%% |    Commands and Environments    |
%% -----------------------------------
\newcommand{\margtodo} {\marginpar{\textbf{\textcolor{kitcolor}{ToDo}}}{}} % used by \todo command
\newcommand{\todo}[1] {{\textbf{\textcolor{kitcolor}{[\margtodo{}#1]}}}{}} % for todo-notes inside the document
\newenvironment{deprecated}{\begin{color}{gray}}{\end{color}}	% for something that you don't want to use anymore

\newcommand{\xcaption}[2]{\caption[#1]{\textbf{#1} #2}}	% nice caption command for short and long description

\newcommand{\xfigure}[5]{\begin{figure}[#1]	% a quick command for including graphics with all necessary variables
\centering
\includegraphics[scale=#2]{./fig/#3}
\xcaption{#4}{#5}
\label{fig:#3}
\end{figure}}

\newcommand{\xfigurerot}[5]{\begin{figure}[#1]	% same as above, only image is rotated
\centering
\includegraphics[angle=270,scale=#2]{./fig/#3}
\xcaption{#4}{#5}
\label{fig:#3}
\end{figure}}

\newcommand{\xtable}[4]{\begin{table}[#1]		% same for tables
\centering
\xcaption{#3}{#4}
\rowcolors{3}{gray!10}{white}
\include{./tab/#2}
\label{tab:#2}
\end{table}}



%% ------------------------------------
%% |    Quantum Mechanics and Math    |
%% ------------------------------------
\newcommand{\ket}[1]{\left|#1\right\rangle}		% \ket{X}  ->  |X>
\newcommand{\bra}[1]{\left\langle#1\right|}		% \bra{X}  ->  <X|
\newcommand{\braket}[2]{\left\langle#1 \middle| #2\right\rangle}		% \braket{X}{Y}  ->  <X|Y>
\newcommand{\bratenket}[3]{\left\langle#1 \middle|\middle| #2 \middle|\middle| #3\right\rangle}		% \bratenket{X}{Y}{Z}  ->  <X|Y|Z>
\newcommand{\anglemean}[1]{\left\langle #1 \right\rangle}		% \anglemean{X}  ->  <X>
\newcommand{\norm}[1]{\left\lVert#1\right\rVert}				% \norm{X}  ->  || X ||

\newcommand{\updownarrows}{\text{\rotatebox[origin=c]{90}{$\rightleftarrows$}}}		% \ket\updownarrows  ->  |↑↓> (this is utf8!)
\newcommand{\downuparrows}{\text{\rotatebox[origin=c]{270}{$\rightleftarrows$}}}	% \ket\updownarrows  ->  |↓↑>
\newcommand{\neswarrows}{\text{\rotatebox[origin=c]{45}{$\rightleftarrows$}}}		% \ket\neswarrows  ->  |↗↙>
\newcommand{\swnearrows}{\text{\rotatebox[origin=c]{225}{$\rightleftarrows$}}}		% \ket\swnearrows  ->  |↙↗>

\newcommand{\cre}{c^\dagger}					% annihalation operator
\newcommand{\anh}{c^{\vphantom{\dagger}}}		% creation operator
\newcommand{\numb}{n^{\vphantom{\dagger}}}		% number operator
